\documentclass[a4paper,11pt]{article}


\usepackage{amssymb}
\usepackage{amsmath}
\usepackage{graphicx}

\usepackage[english]{babel} %English hyphenation
\usepackage[utf8]{inputenc}

%Hyperreferences in the document. (e.g. \ref is clickable)
\usepackage{hyperref}
\usepackage{float}
\usepackage{array}


\title{Advanced Algorithms}
\subtitle{Programming Exercise 1}
\author{Jorik Oostenbrink (4169263) and Thijs Boumans (4214854)}
\date{}

\pagestyle{empty}

\begin{document}
	
\maketitle

\section{Introduction}
The minimum tardiness (scheduling) problem is defined as follows: given a single processor that has to process $n$ jobs $j_1,j_2,\dots,j_n$, where every job $j_i$ has a processing time/length $t_i$ and a due time $d_i$. Find a schedule that minimizes the total tardiness, i.e. $\sum_{i=1}^{n}{\max(0,c_i - d_i)}$, where $c_i$ is the completion time of job $i$.

For programming exercise 1 of the course Advanced Algorithms we had to implement both the exact dynamic programming algorithm given in \cite{exact} and the approximate algorithm (FPTAS) given in \cite{approx}.

\section{Exact Dynamic Programming Algorithm}
In ``A ``PseudoPolynomial'' Algorithm For Sequencing Jobs To Minimize Total Tardiness'' \cite{exact} E.L. Lawler suggested the following method to find an optimal solution to the minimum tardiness problem:

First order the jobs in nondecreasing due time order.
	
\bibliographystyle{unsrt}
\bibliography{mybibfile}

\end{document}

